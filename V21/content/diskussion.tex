\section{Diskussion}
\label{sec:Diskussion}
Zur Kompensation des Magnetfelds der Erde lässt sich sagen, dass die Methode zu relativ groben Fehlern führen könnte, 
da die Apperatur per Hand gedreht wurde.
Dies zeigte sich am Peak, welcher immer noch eine beachtenswerte Breite aufwies.
Dies lässt darauf schließen, dass die Magnetfelder nicht perfekt (anti-)parallel waren.
Ebenfalls war nach Kompensation der vertikalen Komponente immernoch eine deutliche Breite des Peaks zu sehen, wodurch eine
Messunsicherheit in die Messwerte eingebaut wurden.
Anderseits klappte die Durchführung der Variation der Frequenz sehr gut.
Die Frequenzen und Ströme ließen sich sehr gut mittels der Apperatur bestimmen, wie sich in Abbildung \ref{fig:transmissionsspektrum} zeigt.
Dort sind nur kleine Schwankungen zu erkennen, während der lineare Zusammenhang zwischen der Frequenz und der Feldstärke gut kenntlich wird.
Dies wird ebenso durch die kleinen Unsicherheiten in den Fitparametern ersichtlich, womit auch die Landefaktoren und die Kernspins 
genau bestimmen werden konnten.
Dennoch besitzt die Energiedifferenz bei der quadratischen Aufspaltung des Isotops $^{85}\text{Rb}$ eine sehr hohe Unsicherheit, welche größer 
als der Wert selber ist, woraus kein wirklich aussagekräftiges Ergebnis geschlossen werden kann.
Abschließend lässt sich sagen, dass die nötigen Größen (bis auf die Energieaufspaltung) genau bestimmt werden konnten.