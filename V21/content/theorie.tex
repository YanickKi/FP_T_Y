\section{Theorie}
\label{sec:Theorie}
Ziel des Versuchs ist die Bestimmung des Kernspins der Rubidium-Isotope $\ce{^{87}Rb}$ und $\ce{^{85}Rb}$.
Dazu wird das optische Pumpen verwendet, wobei durch Hochfrequenz-Strahlung optisch induzierte nicht-thermische Besetzungen erreicht werden.

\subsection{Atomare Quantenzahlen}
Die Elektronenkonfiguration eines Atoms wird beschrieben durch die Hauptquantenzahl $n$, die Bahndrehimpulsquantenzahl $l$ $(0 \leq l < n)$ und die
magnetische Quantenzahl $m$ $(-l \leq m \leq l)$. 
Rubidium gehört zu der Gruppe der Alkali-Metalle und hat die
Elektronenkonfiguration $[Kr]\,5s^1$. Da die Alkalimetalle nur ein Valenzelektron besitzen ist der Gesamtspin, sowie der Gesamtdrehimpuls
entsprechend dem des Elektrons der $5s$-Schale. Der Kernspin von $\ce{^{85}Rb}$ ergibt sich zu $I = $ \sfrac{5}{2}