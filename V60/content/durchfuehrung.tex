\section{Durchführung}
\label{sec:Durchführung}

\subsection{Vorbereitung}
Ein Diodenlaser, sowie eine Absorbtionszelle mit Rubidium sind auf
auf einem Optischen Tisch mit genügend Abstand voneinander befestigt.
Der Diodenlaser ist mit einem Netzgerät mit Reglern für
Temperatur, Stromeinstellung und weiteren Einstellmöglichkeiten
verbunden, um den für die Rubidiumabsorptionslinien notwendigen Wellenlängenbereich abzudecken.
Mit dem Controller wird der Laser auf die Betriebstemperatur von $\qty{50}{\degreeCelsius}$ erhitzt.
Des weiteren kann am Diodenlaser mithilfe eines sechskantigen
Winkelschraubendrehers das Beugungsgitter horizontal, wie auch vertikal verstellt werden.
Für den weiteren Aufbau sind verschiedene Linsen, ein $50 / 50$-Strahlteiler
und eine CCD-Kamera, um das nicht sichtbare Infrarot-Licht sichtbar zu machen, notwendig.

\subsection{Schwellenstrom}
Im folgenden Versuchsteil wird der Schwellenstrom mithilfe der Lasergranulation bestimmt.
