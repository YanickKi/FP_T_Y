\section{Theorie}
\label{sec:Theorie}
\subsection{Ziel}
\label{subsec:Ziel}
In diesem Versuch wird der Diodenlaser so eingestellt, so dass dieser mit einer Energie abstrahlt, bei welcher Fluoreszenz bei Rubidium beobachtet werden kann. 
Außerdem wird das Absorptionsspektrum von Ribidium aufgenommen.
\subsection{Prinzipielle Funktionsweise eines Diodenlasers}
\label{subsec:prinzipielleFunktionsweise}
Grundlegend für einen Laser ist die Populationsinversion. Eine Populationsinversion liegt z.B. bei einem 2 Niveausystem vor, wenn in diesem System mehr Teilchen
in dem Niveau 2 mit der Energie $E_2$ als in dem Niveau 1 mit der Energie $E_1$, wobei $E_2 > E_1$ gilt, sind.
Wenn jetzt ein Teilchen von Niveau 2 auf Niveau 1 springt, wird aufgrund der Energieerhaltung ein Photon mit der Energie $E_2 - E_1 = \hbar\omega$ abgestrahlt.
Die Abstrahlung eines Photons wird Emission genannt, wobei bei dem Laser die stimulierte Emission genutzt wird, da diese koheräntes Licht abstrahlt, welche bei einem Laser
von großer Bedeutung ist. 
Bei der stimulierten Emission verursacht ein Photon diese, weswegen ein Photon erzeugt wird, welches die gleiche Wellenlänge,
Phase, Polarisation und Ausbreitungsrichtung besitzt. 
Somit ist das emittierte Licht koheränt.
\subsection{Komponenten eines Diodenlasers}
\label{subsec:Komponenten}
\begin{figure}
    \centering
    \includegraphics[width = 0.78\textwidth]{pictures/komponenten.png}
    \caption{Grober Aufbaue eines Diodenlasers\cite{theorie}}
    \label{pic:theorie}
\end{figure}
Ein Diodenlaser (Laser ist ein Akronym für light amplification by stimulated emission of radiation) besteht im wesentlichen aus 3 Komponenten. 
Diese sind das aktive Medium\ref{subsubsec:aktivesMedium}, der interne Resonator\ref{subsubsec:internerResonator} 
und der externe Resonator bzw. das Gitter\ref{subsubsec:externeResonator}.
\subsubsection{Aktive Medium}
\label{subsubsec:aktivesMedium}
In dem aktiven Medium werden Photonen durch p-n-Übergänge erzeugt. P-dotierte Halbleiter sind mit einem Element, welches ein Valenzelektron weniger besitzt, versehen. 
Dieses fehlende Elektron kann als Loch (positiv geladenes Quasiteilchen) behandelt werden.
Das Loch wird durch ein Valenzelektron eines Halbleiteratoms aufgefüllt, wodurch sich das Loch schon bei geringen Temperaturen fortbewegen kann, da an diesem Halbleiteratom ein Loch enstanden ist,
welches wieder aufgefüllt wird. Deshalb wird das Fremdatom auch Akzeptor genannt.
Bei der N-Dotierung wurde dem Halbleiter ein Element hinzugefügt, welches ein Valenzelektron mehr besitzt. Dieses kann sich ebenfalls schon bei geringen Temperaturen frei bewegen,
weshalb dieses Fremadatom Donator genannt wird.
Entscheident bei dem Laser ist der p-n-Übergang, welcher durch Kontakt einer p-dotierten und einer n-dotierten Halbleiterschicht entsteht.
\begin{figure}
    \centering
    \includegraphics{pictures/p-n-uebergang}
    \caption{Diffusion bei Kontakt zwischen einer n- und p-dotiertem Halbleiter\cite{p-n-uebergang}}
    \label{pic:diffusion}
\end{figure}
Dabei diffundieren die überschüssigen Elektron der n-dotierten Schicht in die n-dotierte Schicht, während die überschüssigen Löcher des p-dotierten Halbleiters in die n-dotierte
Schicht diffundieren.
Dadurch fehlen in den jeweiligen Schichten freie bewegliche Ladungsträger und die ortsfesten Dotierungsatome sind nicht mehr elektrisch neutral.  
Somit entsteht eine Raumladungszone mit einer inhomogenen Ladungsverteilung, wodurch eine Spannung erzeugt wird.
Diese Spannung dient für die freien Ladungen als Potentialwall. Das Elektron in Abbildung \ref{pic:kruemmung} kann die Raumladungszone nur überqueren, wenn es die Energie $E_C$ besitzt
, wonach es mit einem Loch rekombinieren kann und Energie, die mindestens der Bandlücke entspricht, in Form von einem Photon abstrahlt.
\begin{figure}
    \centering
    \includegraphics[width = 0.78\textwidth]{pictures/kruemmung.png}
    \caption{Die Krümmung der Baender\cite{kruemmung}}
    \label{pic:kruemmung}
\end{figure}
Wird nun eine externe Spannung gelegt, wobei der negative Pol an die p-dotierte und der positive Pol an die n-dotierte Schicht angelegt wird, wird diese Spannung vermindert,
wodurch die freien Ladungen weniger Energie benötigen, um die Raumladungszone zu überqueren.
Bei umgedrehter Polung würde die Potentialdifferenze erhöht, wodurch die freien Ladungen noch mehr Energie benötigen würden.
\subsubsection{Interner Resonator}
\label{subsubsec:internerResonator}
Der interne Resonator wird durch die gegenüberliegenden Ränder des Kristalls in horizontaler Richtung gebildet, welche eine verschiedene Reflektivität besitzen.
Dadurch kann am richtigen Rand Licht austreten, wobei bei dem anderen Ende kein Licht "verloren geht".
Dadurch kann eine reflektierte Welle durch stimulierte Emission weiter Photon auslösen.
Damit der interne Resonator tatsächlich als Verstärker dient,
muss sich dort eine stehende Welle ausbreiten, für welche die Randbedingung 
\begin{equation}
    \psi \left (kL \right ) \overset{!}{=} 0    \label{eqn:psi}
\end{equation}
gilt. In dieser eindimensionalen Betrachtung ist $\psi$ die Amplitude, $k$ die Wellenzahl und $L$ die Länge des Resonators bzw. Kristalls ist. 
Daraus folgt, dass die Wellen eine Wellenlänge von 
\begin{equation}
    \lambda = 2\frac{L}{n}, \; \; \; n \in \symbb{N} \label{eqn:lambda}
\end{equation}
haben muss.
\subsubsection{Externer Resonator/Gitter}
\label{subsubsec:externeResonator}
Die kollimierende Linse sorgt dafür, dass die Strahlen annähernd kolinear verlaufen, so dass der Laserstrahl gebündelt ist.
An dem Gitter sind die, die duch Interferenz enstehenden, Beugungsmaxima 0. und 1. Ordnung von Relevanz.
Das 0. Maximum wird normal reflektiert, wobei das 1. Maximum bei der richtigen Wellenlänge zurück zum Kristall reflektiert wird, wodurch ein Resonator möglich wird.
Hierbei gilt die Bragg-Bedingung
\begin{equation}
   k \cdot \lambda = 2d \sin \left (  \theta \right ) , \; \; \; k \in \symbb{N} \label{eqn:bragg} ,
\end{equation}
wobei d die Gitterkonstante und $\theta$ der Beugungswinkel ist.
Dieser externe Resonator wird durch die weniger reflektierende Seite des Kristalls und das Gitter gebildet.
Die Funktionsweise ist hierbei dieselbe und die Relationen \ref{eqn:psi} und \ref{eqn:lambda} gelten auch hier, wobei $L$ jetzt nicht mehr die Länge des Kristalls sondern 
der Abstand des Gitters zum Kristall ist.
\subsection{Leistung}
\begin{figure}
    \centering
    \includegraphics[width = 0.78\textwidth]{pictures/gain.png}
    \caption{Die jeweiligen Leistungen der einzelnen Komponenten in Abhängigkeit von der Wellenlänge\cite{theorie}}
    \label{pic:gain}
\end{figure}
Das in der Abbildung \ref{pic:gain} zu sehende Maximum der Kurve des Mediums, lässt sich auf die Bandlücke zurückführen. Diese Photon haben die zu der Wellenlänge
$\lambda_0$ gehörende Energie ab, welche der Bandlücke entspricht. Da Die Bandstruktur jedoch kontinuierlicher und nicht diskreter Natur ist, ist Maximum ein sehr Breites Maximum.
Die Periodizität des internen und externen Resonators lässt sich mit der Bedingung \ref{eqn:lambda} erklären, da nur Wellen mit einer am Rand verschwindenden Ampliute zur Leistung stark beitragen.
Die höhere Frequenz der Kurve von dem externen Resonator in Abbildung \ref{pic:gain} lässt sich ebenfalls mittels \ref{eqn:lambda} erklären, da die 
 
\subsection{Einflüsse auf die Wellenlänge}
\subsubsection{Temperatur}