\section{Diskussion}
\label{sec:Diskussion}
Die Theorieerwartungen und Messergebnisse stimmen sehr gut überein.
Bei der Messung des Schwellenstroms ist eine eindeutige
Stromstärke feststellbar, womit der LED-Bereich des Lasers klar bestimmt werden kann.
Bei der Aufnahme des Transmissionsspektrums sind am Anfang zwei weitere Absorptionslinien sichtbar gewesen.
Davon war eine hinter den erwarteten Absorptionslinien, welche aber durch
einen Modensprung erklärt und durch justieren des Gitters entfernt werden konnte.
Die andere Linie war vor dem Spektum und konnte nicht erklärt werden. Jedoch
konnte durch justieren des Piezostroms auch diese entfernt werden und es enstand das erwartete Absorptionsspektrum.
Die Unterdrückung des Untergrunds durch den $50 / 50 $-Strahlteiler ist auch erfolgreich.
Dies ist zu sehen an der geraden Untergrundlinie, wobei jedoch ein paar kleine Schwankungen auftreten, welche möglicherweise
durch die Detektordioden oder den Aufbau entstehen.
