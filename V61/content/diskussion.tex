\section{Diskussion}
\label{sec:Diskussion}
Zu der Überprüfung der Stabilitätsbedingung \ref{subsec:stabilitaet}lässt sich sagen, dass diese zwar relativ gut überprüft werden konnte, dennoch ist es schwierig diese genau zu überprüfen.
Dies liegt daran, dass es schwierig ist, nah an den Grenzwerten $0$ und $1$ der Stabilitätsbedingung die Spiegel so einzustellen, dass der Laser stabil ist.
Somit ist es schwierig zu erkennen, ob die Einstellung der Spiegel nicht gut genug ist oder die Stabilitätsbedingung nicht mehr erfüllt ist.\\
In dem Abschnitt \ref{subsec:TEM00} wird die Struktur der $\text{TEM}_{00}$-Mode, nämlich die Gaußverteilung, ersichtlich.
Jedoch ließ sich der Aufbau erst nicht einstellen, so dass die $\text{TEM}_{00}$-Mode sichtbar wurde. 
Erst nach dem Austausch eines Spiegels wurde diese sichtbar.
Eine zweite Mode, welche so wenige Knote hat, so dass diese vermessen werden kann, ließ sich trotz Wechsel eines Spiegels nicht erkennen.
Erwartet wurde eine $\text{TEM}_{10}$-Mode, welche ein Intensitätsminimum bei $x=\qty{0}{\metre}$ hat.\\
Anhand der Abbildung \ref{fig:polarisation} wird ersichtlich, dass das Licht eine Polarisation von $\sfrac{\pi}{2}$ hat.
Bei der Wellenlänge in Abschnitt \ref{subsec:wavelength} lassen sich große Abweichungen erkennen. 
Dies wird bereits durch eine hohe Standardabweichung des Mittelwerts deutlich.
Diese hohen Abweichungen lassen sich dadurch erklären, dass das Gitter zum Schirm verdreht war, so dass die Intensitätsmaxima rechts einen deutlich größeren
Abstand von dem nullten Maximum hatten als die links. 
Die Abweichung zum angegeben Wert liegt bei $\eta_{\lambda} \approx \qty{1.30}{\percent}$, was darauf hindeutet, dass sich die Abweichungen herausgemittelt haben.\\
Die Resonatorlängen  konnten in Abschnitt \ref{subsec:longitudal} sehr genau aus den Frequenzabständen $\symup{\Delta}\nu$ berechnet werden.
Die Abweichungen sind in Tabelle \ref{tab:longitudinalabweichung} aufgelistet.
\begin{table}
    \centering
    \caption{Berechnete Abweichung $\eta_L$.}
    \label{tab:longitudinalabweichung}
    \begin{tabular}
      {S[table-format = 3.0] S[table-format = 3.2] S[table-format = 1.2]}
      \toprule
     {$L_{\text{gemessen}} \mathbin{/} \si{\centi\meter}$} & {$L \mathbin{/} \si{\centi\meter}$} & {$\eta_{L} \mathbin{/} \si{\percent}$}\\
      \midrule
       63&   61.81 &  1.88\\
      133&  133.24 &  0.18\\
      153&  152.44 &  0.37\\
      173&  171.31 &  0.98\\
      210&  212.62 &  1.25\\
      \bottomrule
    \end{tabular}
\end{table}
Im Allgmeinen lässt sich sagen, dass es mit fortlaufender Zeit schwieriger wurde den Laser stabil zu halten.
Vermutlich wurde die Qualität der Spiegel durch Staubablagerungen beinflusst. 
Ebenfalls ließen sich Abstände mit einem Messband nicht sehr genau bestimmen.