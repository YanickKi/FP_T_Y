\section{Diskussion}
\label{sec:Diskussion}
Im Allgemeinen lässt sich sagen, dass sich die Analogie zwischen den quantenmechanischen Wasserstoff- bzw. Wasserstoffmolekülmodellen und einem klassischen 
Hohlraumresonator durch diesen Versuch sehr gut verdeutlichen lässt. 
Besonders die Polarplots zeigen dieses Verhalten, da sich dort die Kugelflächenfunktionen mit einem relativ kleinen Aufwand erkennen lassen. Die einzige Ausnahme bildet
der Polarplot \ref{fig:hvarangle27}, welcher die Kugelflächenfunktion $Y_1^0$ darstellen soll. 
Bei einem Winkel von $\theta \approx \ang{135}$ herrscht dort ein Einbruch der Amplitude, wodurch die Unterscheidung zur Kugelflächenfunktion $Y_2^0$ erschwert wird.
Dennoch wird die Struktur dieser Kugelflächenfunktion deutlich.
Ebenso wurde sowohl die Symmetriebrechung durch die Ringe bei dem Wasserstoffatom als auch die Kopplungsstärke in Abhängigkeit des Blendendurchmessers bei
dem Wasserstoffmolekül deutlich.
Ebenso ließ sich die Phasenverschiebung bei dem (anti-)bindenden Zustand veranschaulichen.
Bei dem Festkörper wurde die Aufspaltung der Peaks gut aufgezeigt.
Außerdem muss noch angemerkt werden, dass weitere Gewichte benötigt wurden, um die Lücken zwischen anliegenden Halbkugeln zu schließen, damit Störeinflüsse 
vermieden werden.
Zudem sind bei dem Wassserstoffatom sehr viele Resonanzen samt Phasenverschiebungen, welche in Tabelle
\ref{tab:hphase} aufgezeigt sind, aufgenommen worden. 
Besonders die beiden Phasenverschiebungen bei $\symup{\Delta}\varphi \approx \ang{70}$ können als besonders problematisch angsehen werdenn,
da diese keine klare Tendenz zu einer verschwindenen beziehungsweise zu einer Phasenverschiebung von
$\symup{\Delta}\varphi = \ang{180}$ aufzeigen. Zudem passen diese nicht ganz zu den gemessenen Kugelflächenfunktionen, welche
eine Phasenverschiebung von $\symup{\Delta}\varphi = \ang{0}$ aufweisen sollten.
In Abbildung $\ref{fig:austauschen}$ wird keine klare Veränderung in Abhängigkeit der Länge des Fremdzylinders ersichtlich. 
Dies liegt, wie in dem Abschnitt \ref{subsub:varc} beschrieben, an dem Blendendurchmesser, welcher in der Durchführung 
zu klein gewählt wurde.
Es würde bei einer Störstelle mit einem Zylinder mit dem Durchmesser $d = \qty{37.5}{\milli\meter}$ ein Peak oberhalb jedes Bandes und bei einer Störstelle 
mit einem Zylinder mit dem Durchmesser $d = \qty{62.5}{\milli\meter}$ ein Peak unterhalb jedes Bandes erwartet werden.
Diese neuen Peaks entsprächen einem neuem Zustand.
Bei der erst genannten Störstelle würde es sich um eine p-Dotierung handeln, während bei der zweitgenannten Störstelle eine n-Dotierung vorliegen würde.
Im Allgemeinen lässt sich zu dem eindimensionalen Festkörper sagen, dass dort die Bänder und Bandlücken, besonders in Abbildung \ref{fig:2c1b10},
deutlich wird. Die Bündel von Peaks entsprechen dort den Bändern, während zwischen diesen die Lücken sind.
Wie in Abschnitt \ref{sub:alternating} ersichtlich wird, ist die Differenzierung zwischen alternierenden Zylindern und Blenden wichtig.
Denn bei alternierenden Zylindern können die Peakpositionen und somit Bandpositionen von den einzelnen Rohren erkannt werden, während 
bei alternierenden Blenden Subbänder um die ursprüngliche Peakpositon entstehen.