\section{Diskussion}
\label{sec:Diskussion}
Im Allgemeinen lässt sich sagen, dass sich die Analogie zwischen den quantenmechanischen Wasserstoff- bzw. Wasserstoffmolekülmodellen und einem klassischen 
Hohlraumresonator durch diesen Versuch sehr gut verdeutlichen lässt. 
Besonders die Polarplots zweigen dieses Verhalten, da sich dort die Kugelflächenfunktionen mit einem relativ kleinen Aufwand erkennen lassen. Die einzige Ausnahme bildet
der Polarplot, welcher die Kugelflächenfunktion $Y_1^0$ darstellen soll. 
Bei einem Winkel von $\theta = \ang{135}$ herrscht dort ein Einbruch der Amplitude, wodurch die Unterscheidung zur Kugelflächenfunktion $Y_2^0$ erschwert wird.
Dennoch wird die Struktur dieser Kugelflächenfunktion deutlich.
Ebenso wurde sowohl die Symmetriebrechung durch die Ringe bei dem Wasserstoffatom als auch die Kopplungsstärke in Abhängigkeit des Blendendurchmessers bei
dem Wasserstoffmolekül deutlich.
Ebenso ließ sich die Phasenverschiebung bei dem (anti-)bindenden Zustand varanschauliche.
Bei dem Festkörper wurde die Aufspaltung der Peaks gut aufgezeigt.