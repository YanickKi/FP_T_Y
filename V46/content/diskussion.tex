\section{Diskussion}
\label{sec:Diskussion}
Zu dem Abschnitt \ref{subsec:Feldstärke} lässt sich sagen, dass die maximale Feldstärke mit Hilfe der Hall-Sonde sehr genau bestimmen werden konnte, so dass 
die folgenden Auswertungen mit einer guten Ausgangslage bestritten werden konnten.
In Abschnitt \ref{subsec:winkel} wird auffällig, dass die Differenz der Winkel bei $\lambda = \qty{1.06}{\micro\metre}$ der ersten Probe
sehr groß ist, so dass dieser für die Bestimmung der effektiven Massen ausgelassen werden musste. 
Mögliche Ursachen dafür könnten eine falsche Justage der Apperaturen wie z.B. Halterung für den Interferenzfilter sein.
Ebenfalls lassen sich in Abbildung \ref{fig:prob} Schwankungen erkennen.
Dies könnte auf die Nulltarierung mit dem Oszilloskop zurückgeführt werden, da die Amplitude nie verschwand sondern ein Minimum annahm. 
Somit lies sich nicht immer feststellen, ob bei diesem Winkel das Minimum ist oder nicht.
Ebenfalls könnte die beschädigte Probe (Probe 2) eine Quelle für Ungenauigkeiten sein. 
Diese Probe war nur in Teilen wieder in die Halterung eingeklebt worden, da sie zuvor beschädigt wurde.
Allgemein lässt sich sagen, dass eine Winkelauflösung mit dem Goniometer erzielt werden konnte.