\section{Auswertung}
\label{sec:Auswertung}
\subsection{Bestimmung der maximalen Feldstärke}
\begin{figure}
    \centering
    \includegraphics{build/magn.pdf}
    \caption{Gemessene magnetische Feldstärke in Abhängigkeit von dem Abstand.}
    \label{fig:magn}
\end{figure}
\begin{table}
    \centering
    \caption{Gemessene magnetische Feldstärke in Abhängigkeit von dem Abstand.}
    \label{tab:magn}
    \begin{tabular}
      {S[table-format = 3.0] S[table-format = 3.0]}
      \toprule
      {$d  \mathbin{/} \si{\milli\meter}$} & {$B \mathbin{/} \si{\milli\tesla}$}\\
      \midrule
        85     &     1  \\
        90     &     5  \\
        95     &     23 \\
        100    &     118\\
        105    &     371\\
        107    &     403\\
        109    &     418\\
        111    &     427\\
        113    &     428\\
        115    &     424\\
        117    &     410\\
        119    &     390\\
        120    &     355\\
        125    &     149\\
        130    &     8  \\
        135    &     6  \\
        140    &     1  \\
      \bottomrule
      \end{tabular}
\end{table} %       
\subsection{Bestimmung der Winkeldifferenzen}
\begin{figure}
    \centering
    \includegraphics{build/probe.pdf}
    \caption{Gemessene Winkeldifferenzen in Abhängigkeit von der Wellenlänge.}
    \label{fig:prob}
\end{figure}
\begin{table}
    \centering
    \caption{Gemessene Farraday-Rotation in Abhängigkeit von der Wellenlänge.}
    \label{tab:probe}
    \begin{tabular}
      {S[table-format = 1.3] S[table-format = 2.2] S[table-format = 2.2] S[table-format = 2.2]}
      \toprule
      {$\lambda  \mathbin{/} \si{\micro\meter}$} & {$\symup{\Delta}\theta_1 \mathbin{/} \si{\degree}$} & {$\symup{\Delta}\theta_2 \mathbin{/} \si{\degree}$}
      & {$\symup{\Delta}\theta_3 \mathbin{/} \si{\degree}$}\\
      \midrule
      1.060 &      11.73&       2.36&       2.00\\
      1.290 &       8.46&       3.16&       5.28\\
      1.450 &       6.46&       2.95&       2.82\\
      1.720 &       7.21&       3.13&       5.54\\
      1.960 &       3.08&       3.12&       7.42\\
      2.156 &       3.31&       3.27&       7.86\\
      2.340 &       3.04&       3.65&       7.63\\
      2.510 &      10.53&       4.34&       7.68\\
      2.650 &       2.84&       3.87&       7.72\\
      \bottomrule
      \end{tabular}
\end{table}
\subsection{Berechnung der effektiven Masse der Elektronen}
\begin{figure}
    \centering
    \includegraphics{build/mass2.pdf}
    \caption{Berechneter und gefitteter Rotaionswinkel $\theta_\text{frei}$ in Abhängigkeit $\lambda^2$.}
    \label{fig:mass2}
\end{figure}
Die Parameter ergeben sich für die Probe mit $N=\num{1.2e18}$
\begin{equation*}
    a = \qty{4.18(141)}{\metre}
\end{equation*}