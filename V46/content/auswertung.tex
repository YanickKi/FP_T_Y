\section{Auswertung}
\label{sec:Auswertung}
\subsection{Bestimmung der maximalen Feldstärke}
\begin{figure}
    \centering
    \includegraphics{build/magn.pdf}
    \caption{Gemessene magnetische Feldstärke in Abhängigkeit von dem Abstand.}
    \label{fig:magn}
\end{figure}
\begin{table}
    \centering
    \caption{Gemessene magnetische Feldstärke in Abhängigkeit von dem Abstand.}
    \label{tab:magn}
    \begin{tabular}
      {S[table-format = 3.0] S[table-format = 3.0]}
      \toprule
      {$d  \mathbin{/} \si{\milli\meter}$} & {$B \mathbin{/} \si{\milli\tesla}$}\\
      \midrule
        85     &     1  \\
        90     &     5  \\
        95     &     23 \\
        100    &     118\\
        105    &     371\\
        107    &     403\\
        109    &     418\\
        111    &     427\\
        113    &     428\\
        115    &     424\\
        117    &     410\\
        119    &     390\\
        120    &     355\\
        125    &     149\\
        130    &     8  \\
        135    &     6  \\
        140    &     1  \\
      \bottomrule
      \end{tabular}
\end{table} %       
\subsection{Bestimmung der Winkeldifferenzen}
\begin{figure}
    \centering
    \includegraphics{build/probe.pdf}
    \caption{Gemessene Winkeldifferenzen in Abhängigkeit von der Wellenlänge.}
    \label{fig:prob}
\end{figure}
\begin{table}
    \centering
    \caption{Gemessene Farraday-Rotation in Abhängigkeit von der Wellenlänge.}
    \label{tab:probe}
    \begin{tabular}
      {S[table-format = 1.3] S[table-format = 2.2] S[table-format = 2.2] S[table-format = 2.2]}
      \toprule
      {$\lambda  \mathbin{/} \si{\micro\meter}$} & {$\symup{\Delta}\theta_1 \mathbin{/} \si{\degree}$} & {$\symup{\Delta}\theta_2 \mathbin{/} \si{\degree}$}
      & {$\symup{\Delta}\theta_3 \mathbin{/} \si{\degree}$}\\
      \midrule
      1.060 &      11.73&       2.36&       2.00\\
      1.290 &       8.46&       3.16&       5.28\\
      1.450 &       6.46&       2.95&       2.82\\
      1.720 &       7.21&       3.13&       5.54\\
      1.960 &       3.08&       3.12&       7.42\\
      2.156 &       3.31&       3.27&       7.86\\
      2.340 &       3.04&       3.65&       7.63\\
      2.510 &      10.53&       4.34&       7.68\\
      2.650 &       2.84&       3.87&       7.72\\
      \bottomrule
      \end{tabular}
\end{table}
\subsection{Berechnung der effektiven Masse der Elektronen}
Um die effektive Masse der Elektronen $m*$ zu bestimmen wird die Gleichung \eqref{TEMPLATE} gefittet.
Die Ausgleichsgerade nimmt dabei die Form 
\begin{equation*}
    \theta_\text{frei} = a \lambda^2 + b
\end{equation*}
an, wobei a und b die Fitparameter sind.
Fortan kann die effektive Masse gemäß
\begin{equation}
    m^* = \sqrt{\frac{e_0^3 NB }{8 \pi^2 \epsilon_0 c^3 na}} \label{eqn:mass}
\end{equation}
berechnet werden.
Der dabei bei einer Wellenlänge von $\lambda = \qty{1.55}{\micro\metre}$ verwendete Brechungsindex von GaAs beträgt $n = 3.374$\cite{brechungsindex}.
Dabei wird von dem auf eine Einheitslänge normierte Rotationswinkel der dotierten Probe der auf eine Einheitslänge normierte Rotationswinkel der undotierte Probe abgezogen,
damit nur der Einfluss der freien Elektronen erfasst wird.
\begin{figure}
    \centering
    \includegraphics{build/mass2.pdf}
    \caption{Berechneter und gefitteter Rotaionswinkel $\theta_\text{frei}$ in Abhängigkeit $\lambda^2$ für $
    N=\qty{1.2e26}{\meter\tothe{-3}}, \; d = \qty{1,36}{\milli\meter}$.}
    \label{fig:mass2}
\end{figure}
Die Parameter ergeben sich für die Probe mit $N=\qty{1.2e26}{\meter\tothe{-3}}$ zu
\begin{align*}
    a_1 &= \qty{4181114927968.37(141181194721678)}{\metre\tothe{-3}} \\
    b_1 &= \qty{7.15(6.53)}{\metre\tothe{-1}} \; \text{.}
\end{align*}
Für die Probe mit $N=\qty{2.8e26}{\meter\tothe{-3}}$ können die Parameter zu
\begin{align*}
    a_2 &= \qty{11428079332614.04(406840994387225)}{\metre\tothe{-3}} \\
    b_2 &= \qty{19.70(18.82)}{\metre\tothe{-1}} \; \text{.}
\end{align*}
berechent werden.
Mit Hilfe von Gleichung \eqref{eqn:mass} ergeben sich effektive Massen von 
\begin{align*}
    m^*_1&= \qty{8.91(151)e-32}{\kilo\gram} \\
    m^*_2&= \qty{8.24(147)e-32}{\kilo\gram}
\end{align*} 
\begin{figure}
    \centering
    \includegraphics{build/mass3.pdf}
    \caption{Berechneter und gefitteter Rotaionswinkel $\theta_\text{frei}$ in Abhängigkeit $\lambda^2$ für $
    N=\qty{2.8e26}{\meter\tothe{-3}}, \; d = \qty{1,296}{\milli\meter}$.}
    \label{fig:mass3}
\end{figure}